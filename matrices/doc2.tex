\documentclass[12pt,a4paper]{report}
\usepackage[latin1]{inputenc}
\usepackage{amsmath}
\usepackage{amsfonts}
\usepackage{amssymb}
\usepackage{graphicx}
\usepackage{hyperref}
\usepackage{multicol}
\usepackage[margin=0.4in]{geometry}
\usepackage{karnaugh-map}
\usepackage{amsthm}
\usepackage{mathtools}
\begin{document}
\raggedright{\includegraphics[scale=0.8]{iith.png}} \hspace{12cm}\raggedleft FWC22084\vspace{8mm}\\ 
 \centering \Large \textbf{ASSIGNMENT- MATRICES} \normalsize \vspace{15mm}
\begin{multicols}{2}
 \raggedright \large \textbf{Contents}\normalsize \vspace{2mm}
\begin{itemize}
 \raggedright \item Problem \item Solution\item Construction
\end{itemize}
 \raggedright \large \textbf{Problem:} \normalsize \hspace{2mm}
\\  Find the coordinates of the foot of perpendicular from the point (-1,3) to the line $3x-4y-16=0$.\vspace{2mm}
\\ \raggedright\textbf{solution:} \vspace{2mm}
 \\ \raggedright Let the given line equation be a normal vector to a plane where the given point P(-1,3) lies.The intersection of the plane and line equations gives the coordinates \raggedright of foot of the perpendicular from point p.\vspace{2mm}
  \\The parametric equation of a line  is given by  
\begin{align}
	x = A + \lambda m
\end{align}
  \\The equation of the line perpendicular to normal of a plane 
		and passing through the point $\vec{P}$ is given by 
\begin{align}
	m^{\top}(x-P)  = 0
\end{align}
\\\raggedright Solving both the the eauations gives us the intersection point 'Q'.
\begin{align}   
       Q = A + \frac{ m^{\top}(P-A)}{m^{\top}m}m  
\end{align}      
\\ \raggedright Further solving the equation with  
\\A = $\begin{pmatrix} 0\\-4 \end{pmatrix}$,P = $\begin{pmatrix} -1\\3 \end{pmatrix}$,m = $\begin{pmatrix} 4\\3 \end{pmatrix}$ gives us the foot of the perpendicular
\begin{align}
	Q = \begin{pmatrix} 2.72\\-1.96 \end{pmatrix}
\end{align}
\centering https://github.com/reshma0639/FWC-Assignment-1/blob/main/avrgcc/codes/main.c \vspace{2mm}
\\ \raggedright 4. Now by taking different combination of inputs, check the output being 0 or 1 in regard with the LED off and on respectively. \vspace{2mm}

\centering Table 1: Truth table \vspace{2mm}
\\ \raggedright 5. Cross check the output with the above mentioned truth table for the corresponding input combination(X,Y,Z). \vspace{7mm}\\
\begin{eqnarray*}
FN =(X+Y')(Y'+Z)(X'+Y+Z')
\end{eqnarray*}

\end{multicols}
\end{document}
